\documentclass[draft,linenumbers]{agujournal2018}
\usepackage{apacite}
\usepackage{url} % should fix any errors with URLs in refs.
\usepackage{amsmath}
\usepackage{xcolor}
\usepackage[colorlinks]{hyperref}
\usepackage[colorinlistoftodos]{todonotes}

\draftfalse

\journalname{Space Weather}

\begin{document}

\title{A Test of the Frequency Independence Assumption of Power System Coefficients used in Geomagnetically Induced Current Estimates}

\authors{R.S. Weigel\affil{1} and P.J. Cilliers\affil{2}}

\affiliation{1}{Space Weather Lab, George Mason University}
\affiliation{2}{South African National Space Agency}

\affiliation{1}{4400 University Drive, Fairfax VA 22030}
\affiliation{2}{Hospital Street, Hermanus 7200}

\correspondingauthor{R.S. Weigel}{rweigel@gmu.edu}

\begin{keypoints}
\item Power system coefficients derived from a time series of GIC and geoelectric field measurements are frequency dependent.
\item A GIC model with frequency-dependent system coefficients provides significantly better estimates of GIC than a frequency independent model.
\item Data-derived power system coefficients may differ significantly with those computed using power system configuration information.
\end{keypoints}

\begin{abstract}
  A common assumption used when estimating geomagnetically induced currents (GICs) in a power system given a time series of nearby direct measurements or indirect estimates of the horizontal geoelectric field components $E_x(t)$ and $E_y(t)$ on Earth's surface is that the system is resistive. That is, the approximation $GIC(t)$  $=$ $a_oE_x(t) + b_oE_y(t)$ (Model 1) is used, where $a_o$ and $b_o$ are frequency-independent power system coefficients.  A first test of this assumption is made using GIC measurements in a 187 kV transformer connected to a $\sim$100~km power line in Memanbetsu, Japan and geoelectric field measurements made at the Memanbetsu Magnetic Observatory $\sim$9~km away. A second model (Model 2) is obtained using a frequency domain generalization of Model 1: $GIC(\omega)$ $=$ $a(\omega)E_x(\omega) + b(\omega)E_y(\omega)$. The coefficients $a(\omega)$ and $b(\omega)$ are shown to be frequency dependent and this model provides significantly better estimates of the measured GIC than Model~1. Based on results using a simulated geoelectric field, it is suggested that the measurement-derived frequency dependence of the system coefficients may be explained by spatial variations in the spectrum of the geoelectric field over the spatial extent of the power system. It is also shown that further improvements over Model~2 can be made using frequency-dependent models with the geomagnetic field as an input.
\end{abstract}

\section{Introduction}

Geomagnetically induced currents (GICs) are electric currents in conducting systems that are due to electric fields near the Earth's surface. Time-varying electric and magnetic fields at Earth's surface are driven by time variations in ionospheric currents on time scales of seconds to hours \citep{Ohtani2000} and the movement of Earth's surface relative to slow-varying current systems in Earth's ionosphere that are near stationary relative to the Sun \citep{Stening2013}. Of particular interest are currents induced in electric power systems because they can lead to system degradation, disruption, and failure \citep{Albertson1993,NERC2012,Gaunt2014}. Accurate estimation of GIC magnitudes is important for power system design, retrospective analysis, and mitigation of the impacts of space weather on power systems \citep{Molinski2002,Thomson2010,NERC2012,Gaunt2014}.

A common assumption made in estimates of GIC in an electric power system using either a measured or estimated time series of the horizontal geoelectric field at Earth's surface, $\mathbf{E}(t)$, is that the power system is resistive or quasi-DC in the sense that the relationship $GIC(t) = a_oE_x(t) + b_oE_y(t)$ holds \citep{Albertson1981,Lehtinen1985}. In this case, estimates of the coefficients $a_o$ and $b_o$ can be easily made from the data using linear regression when contemporaneous measurements of $GIC(t)$ and $\mathbf{E}(t)$ are available, or by using information about the connectivity of the power lines and values of the conductor and transformer resistances using DC circuit methods \citep[e.g.,][]{Boteler2014a,Boteler2014b,Boteler2017}. This quasi-DC assumption has been used or stated in many GIC-related works, for example, \citet{Pulkkinen2007,Wik2008,Pirjola2009,Pulkkinen2010,Ngwira2011,Horton2012,Viljanen2012,Overbye2012,Marshall2013,Liu2014,Zheng2014,Watari2015}.

In this work, we use a unique dataset in which time series measurements of $\mathbf{E}$ and the surface horizontal geomagnetic field, $\mathbf{B}$, were made near a site where GIC was being measured in an electric power system.  We first estimate the system coefficients $a_o$ and $b_o$ from data using conventional methods, in which they are modeled as frequency independent. Then, we use a model in which the system coefficients are allowed to be frequency dependent, compute the frequency dependence, and compare this model's ability to represent GIC with the traditional frequency-independent model. Two additional frequency-dependent models are considered that require only the geomagnetic field as an input and they are shown to be significantly better at estimating GIC than the models that use the geoelectric field as an input.

\section{Data}

The 1-second-cadence GIC data used span a total of 35 days across 10 intervals that start and end at midnight Japan Standard Time (JST) and are listed in Table~\ref{intervals}. These data have been presented previously in \citet{Watari2009} and \cite{Watari2015}. These time intervals are those for which GIC data was made available with the exception of three days for which there were obvious problems with either the electric field or GIC measurements. The GIC dataset includes 1~s raw data and 1~Hz low-pass-filtered raw measurements and the results presented are for the 1~Hz low-pass-filtered GIC measurements.

GIC was measured in a grounded neutral point of a Y-connected three-phase transformer connected to the 187 kV bus at the Memanbetsu substation of the Hokkaido Power Co. The 187 kV line extends from the Ashro substation to the Memanbetsu substation, which are both grounded. Memanbetsu is an end point of the 187 kV line.

One-second-cadence electric and magnetic field measurements from Memanbetsu Magnetic Observatory (geodetic latitude: 43.910$^{\circ}$ N, geodetic longitude: 144.189$^{\circ}$ E) were obtained from the Japan Meteorological Agency data portal. The locations of the observatory and substations are shown in Figure~\ref{map}.

Data for the time span of 2006-08-04T15:00Z through 2006-08-08T15:00Z are shown in Figure~\ref{sample}. The GIC data were provided as 1-day files that start at midnight JST and the times shown in Figure~\ref{sample} are Universal Time.

Spikes in the electric field measurements were assumed to be unphysical when the absolute value of a component $i$ of the electric field changed by more than 1~mV/km at time $t$, and the values $E_i(t-2)$ through $E_i(t+4)$ were replaced using linear interpolation of the values at $E_i(t-3)$ and $E_i(t + 5)$. In the sample shown in Figure~\ref{sample}, 21 spikes were identified in $E_x$ and zero in $E_y$. For the time intervals considered, the fraction of interpolated electric field values ranged from 0--0.1\%.

The only modification made to the magnetic field measurements was the replacement of 36 data points on 2006-08-06 with linearly interpolated values. The 36 data points for $B_x$ and $B_y$ have the same value and are much larger than the surrounding values.

The GIC data contains non-physical spikes that are followed by 40-80~s of anomalous values that appear as an exponential decay back to pre-spike values. These spikes were removed by identifying times $t$ when the change in the 1~s absolute value of $GIC(t)$ was greater than $0.05$~Amps and then values in a window $GIC(t-2)$ through $GIC(t+100)$ were replaced using a linear interpolation of the values at $GIC(t-3)$ and $GIC(t + 101)$. In the example interval shown in Figure~\ref{sample}, 493 spikes were removed and approximately 15 spikes that were were not removed by the automated algorithm are still visible. For the time intervals considered, the fraction of interpolated GIC values ranged from 1--4\%.

\section{Models and Methods}
\label{section:Models_and_Methods}

To determine the coefficients $a_o$ and $b_o$ in the frequency independent model, Model~1,

\begin{linenomath*}
  \begin{equation}
    G_o(t) = a_oE_x(t) + b_oE_y(t)
    \label{model1}
  \end{equation}
\end{linenomath*}

\noindent
the matrix equation $\underline{\mathbf{GIC}} = \underline{\mathbf{E}}\cdot\mathbf{p}$ is solved in a least squares sense, where $\underline{\mathbf{GIC}}$ is a zero mean 86400x1 matrix with rows of $GIC(t)$; $\mathbf{p} = [a_o,b_o]^T$, and $\underline{\mathbf{E}}$ is a 86400x2 matrix with rows of $[E_x(t), E_y(t)]$ with zero mean columns. The values of $a_o$ and $b_o$ determined in this way correspond to those that minimize the sum-of-squares of $GIC(t)-G_o(t)$ over a 1-day interval. \citep[][provided the mathematically equivalent analytic equations for solving the matrix equation.]{Pulkkinen2007} The notation $\overline{a}_o$ and $\overline{b}_o$ will be used for the average of the $a_o$ and $b_o$ values computed using 35 1-day-window-segment regressions.

In Model~2, the system coefficients $a(\omega)$ and $b(\omega)$ are frequency dependent (frequency-dependent variables are indicated by explicitly listing their dependence on $\omega$):

\begin{linenomath*}
  \begin{equation}
    G_E(\omega) = a(\omega)E_x(\omega) + b(\omega)E_y(\omega)\,,
    \label{model2}
  \end{equation}
\end{linenomath*}
\noindent
and were estimated using the Fourier transforms of $GIC(t)$ and $\mathbf{E}(t)$; the method of coefficient estimation and the generation of time-domain GIC estimates for this and the following models are given at the end of this section.

Motivated by the results described in the following section, in which Model~2 provides a significantly better representation of GIC than Model~1, we have also considered two additional models that can be used to estimate GIC. In these two models, a direct measurement of the geoelectric field is not used; instead, the driver is the geomagnetic field, which is much more often measured, and as a result, these models may be of practical interest.

In Model~3, previously computed coefficients $a(\omega)$ and $b(\omega)$ from Model~2 are used, but with an electric field $\mathbf{E}'$ computed using the measured magnetic field, $\mathbf{B}$, and a transfer function $\boldsymbol{Z}$: $\mathbf{E}'(\omega) \equiv \boldsymbol{Z}(\omega)\mathbf{B}(\omega)$ and

\setcounter{equation}{2}
\begin{linenomath*}
  \begin{equation}
    G_{E'}(\omega) = a(\omega)E'_x(\omega) + b(\omega)E'_y(\omega)\,,
    \label{model3}
  \end{equation}
\end{linenomath*}

\noindent
where the electric field components are $E'_x(\omega)$ = $Z_{xx}(\omega)B_x(\omega) + Z_{xy}(\omega)B_y(\omega)$ and $E'_y(\omega) = Z_{yx}(\omega)B_x(\omega) + Z_{yy}(\omega)B_y(\omega)$. That is, instead of using the measured electric field directly as in Model~2, a transfer function that relates $\mathbf{E}(\omega)$ to $\mathbf{B}(\omega)$ is used to provide an estimate of the electric field based on the magnetic field.

The fourth model is

\begin{linenomath*}
  \begin{equation}
    G_B(\omega) = z_x(\omega)B_x(\omega) + z_y(\omega)B_y(\omega)\,.
    \label{model4}
  \end{equation}
\end{linenomath*}

\noindent
In this model, the transfer function components $z_x(\omega)$ and $z_y(\omega)$ are determined directly from measurements of $GIC(t)$ and $\mathbf{B}(t)$ and no electric field measurements are used; this model is expected to produce estimates of GIC that are similar to that obtained from Model~3 because they both use the same input measurements.

Estimates of $a(\omega)$ and $b(\omega)$ in Model~2 are made at evaluation frequencies, $\omega_e \equiv 2\pi f_e$, by performing a ordinary least squares regression (OLS). At each evaluation frequency, a regression is performed using a set $G_E(\omega)$, $E_x(\omega)$, and $E_y(\omega)$ values from the DFT (discrete Fourier transform) of 24 hours of 1-s cadence data in a band of frequencies around $\omega_e$. Following the suggestion of \cite{Simpson2005}, the 28 values of $f_e$ start at $0.25$ Hz and decrease by a factor of $\sqrt{2}$ and end at $1/43200$~Hz; if a $f_e$ value computed in this way is not an integer multiple of $1/86400$, the nearest smaller integer multiple is used. The periods associated with the evaluation frequencies are listed in Table~\ref{evaluationperiods}.

The same approach is used for solving for $z_x(\omega_e)$ and $z_y(\omega_e)$ in Equation~\ref{model4}.

For the transfer function $\boldsymbol{Z}$ in $\mathbf{E}(\omega) = \boldsymbol{Z}(\omega)\mathbf{B}(\omega)$ used in Model~3, the $xx$ and $xy$ components of $\boldsymbol{Z}$ at the evaluation frequencies are solved for using the same method used for Model~2 using the equation

\begin{linenomath*}
  \begin{equation*}
    E_x(\omega) = Z_{xx}(\omega)B_x(\omega) + Z_{xy}(\omega)B_{y}(\omega)
  \end{equation*}
\end{linenomath*}

\noindent
and the $yx$ and $yy$ components are solved for using

\begin{linenomath*}
  \begin{equation*}
    E_y(\omega) = Z_{yx}(\omega)B_x(\omega) + Z_{yy}(\omega)B_y(\omega)\,.
  \end{equation*}
\end{linenomath*}

Transfer functions were computed for each of the 1-day segments. The average transfer function magnitude and phase at a given evaluation frequency were obtained by averaging the segment values at that evaluation frequency.

To interpolate the frequency-domain transfer function components that are estimated at the evaluation frequencies onto the original DFT frequency grid (with a spacing of $1/86400$~Hz), their real and imaginary components were linearly interpolated individually. The interpolated transfer function coefficients were set to zero at frequencies higher than $0.375$~Hz and the transfer function estimate at $f=0$ was set to zero before interpolation.

GIC time series were calculated by multiplication of the interpolated transfer function coefficients with the DFT of the 24-hr long and 1~s cadence model inputs and computing the inverse DFT (e.g, for Model~2, $G_E(\omega)=a^I(\omega)E_x(\omega)+b^I(\omega)E_y(\omega)$ was calculated, where the superscript $I$ indicates the interpolated value, and then the inverse DFT of $G_E(\omega)$ was calculated to give the estimate of $G_E(t)$ on a 1~s time grid).

Several modifications to this basic procedure were considered. Robust regression \citep{Egbert1986} with either a Huber weighting using a constant of $1.345$ or bi-square weighting using a constant of $4.685$, both with no hard-cutoff step. The commonly used constants provide 95\% efficiency for Gaussian-distributed errors. Models were also computed using time domain pre-whitening (using a 10th-order Yule-Walker model computed using the model output), time domain tapering (using a Parzen window), and weighting the frequencies around each evaluation frequency using Parzen window weights. When using these additional procedures individually, we find that the conclusions made regarding the differences in each model's prediction ability do not change because the differences in each model's prediction performance are within the calculated uncertainty of the used method. Transfer function coefficients found using these modifications are generally within the error bars of those for the used procedure when the signal-to-noise ratio is above $\sim$3 (corresponding to periods longer than $\sim$30~s); at periods shorter than $\sim$30~s, the electric field measurements are significantly influenced by the instrument response characteristics~\citep{Fujii2015}, aliasing, and the computed values are also expected to be significantly biased due to a low signal-to-noise ratio.

\section{Model Evaluation}
\label{section:Model_Evaluation}

Out-of-sample estimation quality metrics for each model on day $D$ were determined by computing the average model coefficients using the other 34 days of data and then using these average coefficients to generate GIC time series and associated metrics on day $D$. This process was repeated to create a set of 35 out-of-sample prediction metrics for each model. In calculating the metrics, the first and last 10 minutes of the day were excluded to reduce the influence of transients that exist in the estimate at the start of the interval for the amount of time that the causal part of the transfer function impulse responses (computed by Fourier inversion of the interpolated transfer function components) is non-zero and at the end for the amount of time that the acausal part of the impulse responses are non-zero.

The performance of each model in estimating GIC was assessed using three metrics: the prediction efficiency, PE, the mean-squared-error (MSE) ratio, and the signal-to-noise ratio (SN).

The mean-squared-error for model $m$ is
\begin{equation*}
  \mbox{MSE}_m = \frac{1}{N}\sum_{t=t_o}^{t_f} \big(GIC(t)-GIC_m(t)\big)^2 \,,
\end{equation*}
\noindent
where $GIC_m(t)$ is the estimate from model $m$ at time step $t$;  $t_o=600$ and $t_f=85800$, so that sum is over one day of one-second-cadence measurements excluding the first and last 10~minutes. The MSE ratio for model $m$ is MSE$_1$/MSE$_m$.

The prediction efficiency, PE, is the ``Case I'' skill score described by \cite{Murphy1988} and is calculated using
\begin{equation*}
  \mbox{PE}_m = 1 - \frac{\sum_{t=t_o}^{t_f} \big(GIC(t)-GIC_m(t)\big)^2}{\sum_{t=t_o}^{t_f} \big(GIC(t)-\overline{GIC(t)}\big)^2}\,,
\end{equation*}
\noindent
where $\overline{GIC(t)}$ is an average over the same time interval used in the summations. The PE is a measure of the skill of the model with respect to the variance in the predictand and it represents the fraction of the variance in the data that is predicted by the model (when PE~$\ge 0)$.

The average PE and the MSE ratios were computed by averaging the out-of-sample PEs and MSE ratios. The 95\% confidence interval, CI, on the PE and MSE ratios were calculated using the bootstrap method \citep{Zoubir1998}: 1000 random samples (with replacement) of 17 of the 35 out-of-sample prediction metrics were used to compute the CIs. The typical difference between the limits when the confidence intervals were computed assuming the values were Gaussian-distributed was $\sim$2\%.

The frequency dependent signal-to-noise ratio for each 1-day segment at a given evaluation frequency is the ratio of of the smoothed power spectrum value of $GIC(\omega_e)$ to the smoothed power spectrum value of the prediction error at that evaluation frequency. A smoothed spectrum was obtained by averaging the raw spectrum in the same frequency bands used for the regression described in Section~\ref{section:Models_and_Methods}. The average signal-to-noise ratio at a given evaluation frequency is the average of the 35 out-of-sample signal-to-noise ratios at that frequency.

\section{Results}
\label{results}

The PE and MSE ratios are shown in Table~\ref{resultstable}. Figure~\ref{predictions} shows the predictions for each model in a 1-day interval selected because the value of the prediction efficiency for each model was in the same order as that shown in Table~\ref{resultstable}, i.e., Model~1 has the lowest PE and Model~4 has the highest. The large errors visible at the start and end of the time intervals are due to the transients discussed in Section~\ref{section:Model_Evaluation} and the first and last 10~minutes of each interval were omitted from metrics calculations.

\begin{table}
  \caption{Out-of-sample model performance metrics for Model $m$. MSE$_1$/MSE$_m$ is the MSE ratio of Model~1 with respect to that of Model $m$. The PE and MSE ratios are the average of 35 values and the confidence interval, CI, was determined from 1000 bootstrap samples.}
  \centering
  \begin{tabular}{l c c c c}
    \hline
    $m$\hspace{1em} Model & PE & 95\% CI & MSE$_1$/MSE$_m$ & 95\% CI\\
    \hline
    2\hspace{1em} $G_E(t) = a(\omega)E_x(t) + b(\omega)E_y(t)$ & 0.60 & [0.54, 0.65] & 1.8 & [1.6, 2.0]\\
    3\hspace{1em} $G_{E'}(t) = a(\omega)E'_x(t) + b(\omega)E'_y(t)$ & 0.78 & [0.76, 0.80] & 3.1 & [2.8, 3.4]\\
    4\hspace{1em} $G_{B}(t) = z_xB_x(t) + z_yB_y(t)$ & 0.83 & [0.80, 0.85] & 4.6 & [3.8, 5.3]\\
    \hline
  \end{tabular}
  \label{resultstable}
\end{table}

The primary conclusions of this work follow from the results shown in Table~\ref{resultstable}: the frequency dependent model, Model~2, is significantly better at estimating GIC than the frequency independent model, Model~1; Model~3, which uses an estimate of the electric field based on magnetic field measurements, provides significantly better estimates of GIC than Model~2; Model~4, in which a transfer function that connects GIC to $\mathbf{B}$ was derived directly from the data, provides better estimates of GIC than Model~3.

To determine the statistical significance of the PE differences, $10^5$ bootstrap averages of the segment PEs of each model were created and their differences were computed. From this, the null hypothesis that the PE of Model~2 is equal to that of Model~1 can be rejected at a significance level of less than $10^{-5}$ because no bootstrap average PE difference from Model~2 was found to be lower than that for Model~1. The null hypothesis that the PE of Model~3 is equal to that of Model~2 can also be rejected at the same significance level. The null hypothesis that the PE of Model~4 is equal to that of Model~3 can be rejected at a significance level of $10^{-2}$.

Figure~\ref{SN} shows the average signal-to-noise (i.e., signal to prediction error) ratios as a function of period for the four models. At each evaluation frequency, the SN ratio is the ratio of the average of the out-of-sample SN ratios and the error bars represent a 95\% CIs based on 1000 bootstrap samples of the segment SN ratios. Consistent with the results in Table~\ref{resultstable}, the SN ratio for Model~1 is lower than that for Models~2-4 at all frequencies. However, ordering of the SN ratio for Models~2-4 is dependent on period.

Below $600$~s, Model~2 has the largest SN ratio. This result is somewhat visible in the sample interval shown in Figure~\ref{predictions} where the high-frequency fluctuations in the prediction error for Model~2 are the smallest among the models. Below $600$~s, the predictive ability of each model is in the order that may be expected based on the data used to derive the model: Model~2, which uses direct measurements of the electric field, has a larger SN ratio than Model~3, which uses an indirect estimate of the electric field. Model~4 has a slightly larger SN than Model~3, and this is expected because they both use the same input of the magnetic field, but Model~3 has an additional source of uncertainty from the data-estimated coefficients in $\mathbf{Z}$.

Above $600$~s, the ordering of the SN is not consistent with these expectations and is possibly explained by two factors: (1) the measured GIC is based on an integrated electric field along the power line, and the electric field used in Model~2 is a point measurement made 9~km away from the line - it follows that there is an implicit assumption in the model that the electric field is uniform over the length of the line on which GIC is measured; and (2) the error associated with this assumption has less of an influence on the errors in a model that uses point measurements of the magnetic field. Issues related to this are discussed in the following section.

Above $10,000$~s (2.8~hrs), the SN ratio ordering is the same as the ordering of the PE and MSE ratios shown in Table~\ref{resultstable}, with Model~1 having the lowest and Model~4 having the largest SN ratio.

The average coefficients and 95\% CI for Model 1 are $\overline{a}_o = 73.0$ $[62,83]$ A/(V/km) and $\overline{b}_o = -72.0$ $[-98,-45]$ A/(V/km) and the distribution of the coefficients for the segments is shown in Figure~\ref{histogram}. (Results are similar if 1-minute averages are used; \cite{Watari2015} found $a_o=38.1$ A/(V/km) and $b_o=-7.4$ A/(V/km) for the interval 2006-12-14 to 2006-12-15 JST time interval). 

The values of $\overline{a}_o = 73$ A/(V/km) and $\overline{b}_o = -72$ A/(V/km) and Model~1 imply that for a given electric field magnitude, GIC is largest at the measurement site when the electric field points 45$^{\circ}$ west of north, which is approximately perpendicular to the 187~kV line shown in Figure~\ref{map} that runs in the northeast direction. \cite{Pulkkinen2010} used magnetic field data at Memanbetsu and GIC data for a 3-day interval starting on 2005-12-26 and found a $b_o/a_o$ ratio of $-1.9/50$, corresponding to a maximum GIC when the electric field angle is 2$^{\circ}$ west of north; The $b_o/a_o = -7.4/38.1$ ratio documented in \cite{Watari2015} using data from a 2-day interval starting on 2006-12-14 that corresponds to an angle of 11$^{\circ}$ west of north.

%The values of $a_o$ and $b_o$ are highly sensitive to the method number of points used in the regression. The values of $\overline{a}_o = 73$ A/(V/km) and $\overline{b}_o = -72$ A/(V/km) were found using 86400 points to compute 35 pairs of $a_o$ and $b_o$ values. If 864 points are used to compute 3500 pairs of $a_o$ and $b_o$ values, $\overline{a}_o=+25.7$ and $\overline{b}_o=+15.2$ A/(V/km) and Model~1 predicts maximum GIC when the electric field is approximately 30 degrees east of north, in the approximate direction of the 187 kV line. 

%However, as shown in Figure~\ref{map}, the 187 kV line which terminates at the Memanbetsu substation runs in the northeast direction, and hence the dominant contribution to the measured GIC is expected be from currents flowing in this line. If the electric field is the same at all points on the line, GIC in the grounded transformers at Memanbetsu substation should be the largest when the electric field points in the northeast direction.

The average coefficients for Models~2 and~4 are shown in Figure~\ref{Z}--\ref{Phi}. The error bars at each period correspond to the standard error of the 35 values used to compute the average.

Figure~\ref{Z} shows that the frequency domain transfer functions for Model~2 vary by an order of magnitude on timescales of 60~s to 12~h (that for Model~1 is constant by definition). For this system, GIC is more sensitive to $E_y$ than $E_x$ due to $b(\omega) > a(\omega)$, but $E_y(t)$ has less variability than $E_x(t)$ - the variance in $E_x(t)$ is $\sim$5x larger than $E_y(t)$ over the entire dataset (this is also visible in Figure~\ref{sample}). 

The ratio of $|\overline{b}_o|/|\overline{a}_o|$ is approximately 1 for Model~1 and this can be compared with $|b(\omega)|/|a(\omega)|$ calculated using the lines for $|a(\omega)|$ and $|b(\omega)|$ shown in Figure~\ref{Z}. The ratio varies from 2.7 at $T=30$~s to 1.6 at $T=12$~h. 

Figure~\ref{Phi} shows the transfer function coefficient phase angles. For Model~1, the phase angle is 0$^{\circ}$ for positive coefficients and 180$^{\circ}$ for negative coefficients by definition. The phases for $a(\omega)$ and $b(\omega)$ are frequency dependent. At $T=12$~h, the phases of $a(\omega)$ and $b(\omega)$ differ by 180$^{\circ}$ and the associated signs are consistent with those of $\overline{a}_o > 0$ and $\overline{b}_o < 0$. At $T=12$~h, the values of $b(\omega)$ and $a(\omega)$ and Model~2 correspond to an electric field of 60$^{\circ}$ west of north being associated with the largest GIC.

Both Model~1 and~2 predict a maximum GIC when the electric field angle is significantly different from the angle of the 187~kV line. As discussed in the following section, this can be caused by other parts of the power network (the details of which are unavailable) making significant contributions to GIC and differences in ground conductivity in parts of the network that contribute to GIC. 

Figure~\ref{Z_MT} and~\ref{Phi_MT} show the magnitude and phase for the coefficients of $\mathbf{Z}$ in Model~3. Note that transfer function coefficients shown in Figure~\ref{Z_MT} differ significantly from that of a layered slab conductivity model for which $Z_{xy}=-Z_{yx}$ and $Z_{xx}=Z_{yy}=0$. The electric and magnetic field data from Memanbetsu, Japan were also used by \cite{Fujii2015} to compute $\mathbf{Z}$. When using the \cite{Fujii2015} transfer function for Model~3, the PE was significantly lower (0.58 vs. 0.78). Although not critical to the conclusions in this work, a preliminary investigation suggests several explanations. We have found that when using continuous one-month intervals prior to 2006 to compute $\mathbf{Z}$, our calculated transfer function coefficients match the \cite{Fujii2015} coefficients to within the error bars except below periods of 100~s, which is likely explained by the issues with the electric field response in the 1-second-cadence measurements discussed in \cite{Fujii2015} and the fact that we did not use a remote reference in our calculations. However, using one-month intervals after 2006 (January through May tested) our calculated transfer function coefficient magnitudes differ significantly from that in \cite{Fujii2015} - they are shifted downward by approximately 10-50\%. In addition, we have found that the estimated transfer function is sensitive to whether or not segment averaging was used. Due to the large gaps between and the varying length of continuous segments of data used in this work, transfer functions were computed using one-day segments of data and then averaged. We have found that if a continuous span of measurements over 21 days is used to compute a single transfer function, the transfer function coefficient magnitudes are up to 25\% larger than if 21 transfer functions were computed and then averaged. This result is unexpected, and results from extensive verification tests of our MT calculations along with the fact that this effect has observed by both authors using two different MT calculation codes on different data sets (the other code used was LEMI MT V1.4), suggests the explanation is not due to an error in the code used to compute the transfer functions in this work. This result suggests that stack averaging can be a significant source of transfer function bias and we expect that the explanation involves variations in the signal-to-noise between segments.

\section{Discussion}
\label{discussion}

There are at least three possible explanations a frequency dependence in data-derived system coefficients: (1) the quasi-DC approximation is invalid, (2) it is an artifact of the measurement process, and (3) the electric field is not spatially and spectrally constant across the region of the power system.

(1) When the quasi-DC approximation is used, \cite{Albertson1981} is usually cited; their justification for the quasi-DC approximation is given in an Appendix as ``An analysis based on Carson's equations reveals that transmission lines can be modeled for GIC determination by using their positive sequence resistance values with a small correction factor to account for skin and proximity effects. The correction factor varies from 0.95 to 1.0 depending on conductor size.'' (Carson's equations are equations for the impedance associated with a wire above an infinite and uniform half-space conducting slab [\cite{Carson1926,Grigsby2007}] and positive sequence resistance values are the per phase line resistances in a balanced 3-phase power system.)

\cite{Lehtinen1985} justifies the use of the quasi-DC approximation using the argument that GIC amplitudes are significant at frequencies that are much lower than the 50~Hz power line frequency. Clarified, this is essentially a statement that the reactance is proportional to frequency and although the reactance is typically $\sim$5 times larger than the resistance at 50~Hz \citep{Purchala2005}, the reactance will be $\sim$10 times smaller or less than the resistance at GIC frequencies of $\sim$1~Hz and lower. Given these arguments, the quasi-DC approximation is justified, and an explanation involving a violation of the assumption of a uniform half-space conducting slab Earth using a conductivity that varies with depth to more accurately model skin an proximity effects also seems unlikely to explain the results as it would need to account for at least an order of magnitude of reactance. (However, \cite{Boteler1994} noted that for certain systems, the $L/R$ time constant of the power network may be large enough to warrant inclusion in GIC flow calculations; this topic has more recently been pursued by \cite{Oyedokun2013a}.)

(2) A second consideration is that artifacts of the electric field and GIC measurements used in Model~2 cause the observed frequency dependence. However, the GIC instrument (HIOKI 9279 current sensor) is frequency independent to within $\pm 0.2$\% in this period range and the response characteristics of the magnetic and electric field instruments are essentially frequency independent relative to the observed transfer function frequency dependence for periods in the range of $30-10^{5}$~s (\cite{Oowada1998}; Fujii 2019 personal communication.)

(3) The coefficients in Model~1 are determined assuming that the electric field is spatially uniform at all points in the system. The exact calculation of GICs in a power system under the quasi-DC assumption requires knowledge of the electric field at all points of the system, and simulations of GIC flow typically start with an assumed electric field along the lines to calculate input voltages \citep{Albertson1981,Lehtinen1985, Pirjola2009}. The electric fields are then used along with the DC system component resistances and their connectivity to calculate GIC in the different parts of the system. In both Model~1 and Model~2, the input electric field is typically assumed to be spatially uniform along the line in which GIC was measured so that the potential difference is directly proportional to the electric field. Spatial variations in the electric field measured on Earth's surface can be due to horizontal variations in the incident electric field and horizontal variations in sub-surface conductivity, including those at land-ocean interfaces. If a spatial variation exists along the line in which GIC was measured, the coefficients derived from measurements in both models may differ significantly from those calculated using system configuration and resistance information. However, the data-derived coefficients will not have a frequency dependence that results only from the variation in the direction of a static electric field along the line.

The second assumption of Model~1 is that the spectrum of the electric field is uniform. Violation of this assumption can lead to an apparent violation of the quasi-static approximation. Consider an electric field and GIC that are measured at one end of a long line segment with the electric field being non-zero only at a frequency of $\omega_1$. If the electric field is spectrally uniform along the full line segment, then from the quasi-static approximation it follows that the measured GIC will also only be non-zero at a frequency of $\omega_1$. If the electric field at the other end of the line is non-zero only at a frequency of $\omega_2$, the observed GIC and electric field will appear to be inconsistent with the quasi-static approximation because GIC will have non-zero variations at both $\omega_1$ and $\omega_2$.

To test the effect of different points on the line having different electric field spectra on the coefficient estimates, we created a simulated GIC signal using electric field measurements made in Memanbetsu (indicated with a superscript M) and Kakioka (superscript K), Japan:

\begin{equation}
  G_s(t)=a_o\big[(1-r)E^{\mbox{\tiny M}}_x(t) + rE^{\mbox{\tiny K}}_x(t)\big]+b_o\big[(1-r)E^{\mbox{\tiny M}}_y(t) + rE^{\mbox{\tiny K}}_y(t)\big]
\end{equation}

\noindent
with $a_o=1$, $b_o=-1$ and $r=0.2$. $G_s$ represents a simulated GIC in which 20\% of the line has a different electric field. (The zero-lag cross correlation is -0.40 for $E^{\mbox{\tiny M}}_x(t)$ and $E^{\mbox{\tiny K}}_x(t)$ and 0.64 for $E^{\mbox{\tiny M}}_y(t)$ and $E^{\mbox{\tiny K}}_y(t)$). The model equation
\begin{equation}
  G_s(\omega)=a(\omega)E^{\mbox{\tiny M}}_x(\omega)+b(\omega)E^{\mbox{\tiny M}}_y(\omega)
\end{equation}
\noindent
was then used to compute $a(\omega)$ and $b(\omega)$ and they were both found to exhibit a frequency dependence similar to that shown in Figure~\ref{Z} in that they both have an increase in magnitude with period, although in this experiment the increase was a factor 2-6 for periods in the range of $2$~s to $12$~hr.

Although this proof-of-concept simulation shows how variations in the frequency content of the geoelectric field on the line in which GIC is measured can cause the data-estimated system coefficients to have a frequency dependence, it is very crude as it uses several approximations that were necessitated by the availability of data. The distance from Kakioka to Memanbetsu is $\sim$1400~km, which is $\sim$14x that of the length of the power line ($\sim$100~km) on which GIC was measured, and the correlations between the electric field from the two sites are likely to be lower than the fields measured at points separated by a distance closer to the length of the power line. In addition, the discontinuous change in the electric field implicit in the model implies that there would be a part of the line that can have a non-zero net charge and thus is not physically realizable for typical conductors. A more realistic simulation would use electric field measurements that were both near a line on which GIC was measured and separated from each other by a distance that is comparable to the length of the line. 

In the discussion of (3) above, we have attempted to address independently the contribution of spatial variations in the electric field magnitude and direction and spatial variations in the electric field spectra, but generally these variations do not occur independently as they are both caused by horizontal variations in the ground conductivity. A time-varying ionospheric current can also cause spatial and spectral variations in the electric field magnitude \citep{Boteler1998}. Thus, the finding in Section~\ref{results} that the angle of the electric field that produces maximum GIC differs significantly from the angle of the 187~kV power line for both Model~1 and Model~2 may be consistent with the frequency dependence of $a$ and $b$ found in Model~2. However, testing this claim would require a full network analysis, knowledge of the ground conductivity throughout the network, and accounting for the driving ionospheric currents.

%(\cite{Uyeshima2001} has shown significant variations in the impedance tensor $\mathb{Z}$ on Hokkaido Island)
 
\section{Conclusions}

The primary implications of the results presented are (1) significantly improved estimates of GICs using data-derived system coefficients $a$ and $b$ can be made if they are allowed to be frequency dependent and (2) a direct comparison of data-derived $a$ and $b$ coefficients and those computed using power system configuration information will be problematic if the data are not consistent with the assumptions of Equation~\ref{model1}.

Although the use of frequency independent $a$ and $b$ coefficients estimated from data give a reasonable estimates of GIC, it is not known how much data-estimated coefficient values will differ from those directly calculated. (Due to the limited availability of details of the power system and the ground conductivity along the line, we were not able to make this comparison.) Works that use Model~1 and generally note a high degree of visual consistency between the measured values and predictions using this model (e.g., the top panel of Figure~\ref{predictions}). However, we have shown using contemporaneous GIC and electric field measurements that a significantly better estimation of GIC can be obtained by using a frequency dependent model of the form of Models~2 and 3.

%The out-of-sample prediction quality reported here for Model~1 (a PE of 0.35 and CC of 0.64). If only a few days of data are used for their estimation, the out-of-sample prediction quality using all data is very poor due to the high degree of uncertainty of in their estimates \todo{Give example}. As a result, predictions of GIC using Model~1 in the literature using only a few days of data are expected to be significantly worse if out-of-sample predictions had been used.

When GIC and magnetic field measurements are available, a model of the form of Model~4 can be used. For this particular data set, we have found that such a model provides the best estimates of GIC in terms of the PE, MSE, and signal-to-error metrics. Local ground magnetic field estimates along with GIC measurements could be used to derive a transfer function model that is updated when the system configuration changes and this model could be used to estimate GIC under extreme or simulated geomagnetic conditions. A key advantage of this approach is that long-term geomagnetic field measurements are generally easier to make and more readily available than geoelectric field measurements. A disadvantage is the transfer function coefficients of Model~4 are difficult to interpret physically because they contain components of the transfer functions that connect the electric field to the magnetic field and GIC to the electric field.

\acknowledgments
Magnetic and electric field measurements made at the Kakioka Magnetic Observatory were obtained from the Japan Meteorological Agency Data Portal \url{http://www.kakioka-jma.go.jp/obsdata/metadata/en}. The geomagnetic coordinates in Figure~\ref{map} were computed using code from \url{https://www.ngdc.noaa.gov/geomag/geom_util/apex.shtml}. The \cite{Fujii2015} transfer function was obtained from the IRIS DMC EMTF database [\cite{Kelbert2011}]. We thank the Hokkaido Electric Power Co. for the GIC measurements at the Memanbetsu substation and Shinichi Watari for communicating and preparing the GIC data and providing feedback on the manuscript. We also thank Ikuko Fujii for providing clarifications on the geoelectric instrument response, Trevor Gaunt and Emanuel Benabeau for review and comments on part (1) of Section~\ref{discussion}, and Josh Rigler for substantial feedback and review.
%\clearpage

\begin{figure}[h]
  \centering
  \includegraphics[width=\textwidth]{figures/map.pdf}
  \caption{Locations of measurement sites for GIC, at the Memanbetsu substation, and the geoelectric and geomagnetic field measurement site at Memanbetsu Magnetic Observatory. The red line corresponds to the path of the 187 kV power line. The distance from the observatory to the Memanbetsu substation is $\sim$9 km and dotted lines are Apex geomagnetic latitudes \citep{Richmond1995} in 2015.}
  \label{map}
\end{figure}

\begin{figure}[h]
  \centering
  \includegraphics[width=\textwidth]{figures/plot_raw_All_20060805-v1.pdf}
  \caption{(Top) Geomagnetic field measured at the Memanbetsu Magnetic Observatory (MMB). (Middle) Geoelectric field measured at the Memanbetsu Magnetic Observatory (MMB). (Bottom) GIC measurements at Memanbetsu substation. The subscripts $x$ and $y$ correspond to Geographic North and East, respectively, the date labels are for Universal Time, and the average has been subtracted from each time series.}
  \label{sample}
\end{figure}

\begin{figure}[h]
  \centering
  \includegraphics[width=\textwidth]{figures/plot_GIC_predictions-MeanModel-2006-08-06-v0-o0.pdf}
  \caption{Out-of-sample model predictions (top) and predictions errors for each model in a selected 1-day interval. The scales for the error time series differ from that for the top panel so that the error amplitudes are scaled up by a factor of 2.}
  \label{predictions}
\end{figure}

\begin{figure}[h]
  \centering
  \includegraphics[width=\textwidth]{figures/plot_model_summary_SN-options-1-v0-o0.pdf}
  \caption{Out-of-sample signal-to-noise (signal to prediction error) ratios at the evaluation frequencies for the four models.}
  \label{SN}
\end{figure}

\begin{figure}[h]
  \centering
  \includegraphics[width=\textwidth]{figures/plot_model_summary_aobo_histograms-options-1-v0-o0.pdf}
  \caption{Distribution of coefficients in Model~1. Each set of coefficients was computed 35 times using one day of 1-second cadence data. The circle or dot on the horizontal axis indicates the average of a histogram and $\overline{a}_o$ = 73 A/(V/km) and $\overline{b}_o$ = -72 A/(V/km).}
  \label{histogram}
\end{figure}

\begin{figure}[h]
  \centering
  \includegraphics[width=\textwidth]{figures/plot_model_summary_Z-options-1-v0-o0.pdf}
  \caption{Frequency domain transfer functions for the coefficients in Models 1, 2, and 4. The frequency domain transfer functions for $a_o$ and $b_o$ in Model~1 are constant and equal to $a_o$ and $b_o$.}
  \label{Z}
  %\end{figure}

  \vspace{4em}

  %\begin{figure}[h]
  \centering
  \includegraphics[width=\textwidth]{figures/plot_model_summary_Phi-options-1-v1-o0.pdf}
  \caption{Frequency domain phase values for the coefficients in Models 1, 2, and 4. The phase for $a_o$ and $b_o$ are constant and either 0$^{\circ}$ or 180$^{\circ}$ depending on the sign of their value, with positive values having a phase of 0$^{\circ}$.}
  \label{Phi}
\end{figure}


\begin{figure}[h]
  \centering
  \includegraphics[width=\textwidth]{figures/plot_model_summary_Z_MT-options-1-v0-o0.pdf}
  \caption{Frequency domain transfer functions for the $\mathbf{Z}$ coefficients used in Model~3.}
  \label{Z_MT}
  %\end{figure}

  \vspace{4em}

  %\begin{figure}[h]
  \centering
  \includegraphics[width=\textwidth]{figures/plot_model_summary_Phi_MT-options-1-v1-o0.pdf}
  \caption{Frequency domain phase values for the $\mathbf{Z}$ coefficients used in Model~3.}
  \label{Phi_MT}
\end{figure}

\clearpage

\section{Appendix}

\setcounter{table}{0}
\renewcommand{\thetable}{A\arabic{table}}

\begin{table}[h]
  \caption{Date intervals used for analysis}
  \centering
  \begin{tabular}{l l}
    \hline
    Start Date & End Date\\
    \hline
    2006-04-03 & 2006-04-10\\
    2006-07-25 & 2006-07-29\\
    2006-08-05 & 2006-08-08\\
    2006-08-19 & 2006-08-21\\
    2006-11-07 & 2006-11-12\\
    2006-11-28 & 2006-11-29\\
    2006-12-01 & 2006-12-14\\
    2006-12-15 & 2006-12-15\\
    2007-11-18 & 2007-11-20\\
    2007-11-22 & 2007-11-22\\
    \hline
  \end{tabular}
  \label{intervals}
\end{table}

\begin{table}
  \caption{Periods associated with evaluation frequency bands used for regression and smoothing spectra. \# is the evaluation frequency number, $N$ is the number of DFT points in the band, and $T_e\equiv 1/f_e$. The band range is from $T_l$ through $T_h$.}
  \centering
  \begin{tabular}{l l l l l}
    \hline \\
    \# & $N$ & $T_e$ [s] & $T_l$ [s] & $T_h$ [s] \\
    \hline \\
    1 & 2 & 43200 & 28800 & 86400 \\
    2 & 2 & 28800 & 21600 & 43200 \\
    3 & 3 & 21600 & 14400 & 43200 \\
    4 & 4 & 14400 & 9600 & 28800 \\
    5 & 5 & 10800 & 7200 & 21600 \\
    6 & 6 & 7854.5 & 5400 & 14400 \\
    7 & 8 & 5760 & 3927.3 & 10800 \\
    8 & 12 & 3927.3 & 2618.2 & 7854.5 \\
    9 & 16 & 2880 & 1920 & 5760 \\
    10 & 22 & 2009.3 & 1350 & 3927.3 \\
    11 & 31 & 1440 & 960 & 2880 \\
    12 & 43 & 1016.5 & 680.3 & 2009.3 \\
    13 & 61 & 720 & 480 & 1440 \\
    14 & 85 & 511.2 & 341.5 & 1016.5 \\
    15 & 120 & 361.5 & 241.3 & 720 \\
    16 & 170 & 255.6 & 170.4 & 511.2 \\
    17 & 240 & 180.8 & 120.5 & 361.5 \\
    18 & 338 & 128 & 85.4 & 255.6 \\
    19 & 478 & 90.5 & 60.3 & 180.8 \\
    20 & 676 & 64 & 42.7 & 128 \\
    21 & 956 & 45.2 & 30.2 & 90.5 \\
    22 & 1351 & 32 & 21.3 & 64 \\
    23 & 1910 & 22.6 & 15.1 & 45.2 \\
    24 & 2701 & 16 & 10.7 & 32 \\
    25 & 3819 & 11.3 & 7.5 & 22.6 \\
    26 & 5401 & 8 & 5.3 & 16 \\
    27 & 7638 & 5.7 & 3.8 & 11.3 \\
    28 & 10801 & 4 & 2.7 & 8 \\
    \hline \\
  \end{tabular}
  \label{evaluationperiods}
\end{table}

\clearpage

\bibliography{paper.bib}

\end{document}
